\documentclass[12pt]{ctexart}
\usepackage{amsmath}
\usepackage{booktabs}
\usepackage{pifont}
\usepackage{enumitem}
\usepackage{esint}
\usepackage{fancyhdr}
\usepackage{float}
\usepackage{fontspec}
\usepackage{geometry}
\usepackage{hyperref}
\usepackage{import}
\usepackage{indentfirst}
\usepackage{lastpage}
\usepackage{nicematrix}
\usepackage{physics}
\usepackage{setspace}
\usepackage{subcaption}
\usepackage{tikz}
\usetikzlibrary{shapes}
\usepackage{titlesec}
\usepackage{titletoc}
\usepackage{wrapfig}
\usepackage{xcolor}
\usepackage{xparse}
\usepackage{xstring}
\usepackage{zhnumber}
\geometry{a4paper,top=2cm,bottom=3.2cm,inner=2.5cm,outer=3.2cm}
\setmonofont{Fira Code}
\setCJKmainfont{Noto Serif CJK SC}
\setCJKsansfont{Noto Sans CJK SC}
\setCJKmonofont{Fira Code}
\DeclareMathOperator{\ee}{\mathrm{e}}
\DeclareMathOperator{\ii}{\mathrm{i}}
\DeclareMathOperator{\jj}{\mathrm{jj}}
\renewcommand{\thesection}{\zhnum{section}}
\titleformat{\section}{\normalsize}{\bfseries\thesection、}{0.25em}{}
\titleformat{\subsection}{\normalsize}{\bfseries\arabic{subsection}.}{1em}{}
\titleformat{\subsubsection}{\normalsize}{\bfseries\hspace{2em}(\arabic{subsubsection})}{.5em}{}
\allowdisplaybreaks[4]
\setstretch{1.5}
\setlength{\parskip}{0.4em}
\everymath{\displaystyle}
\NewDocumentCommand{\options}{O{4}mmmm}{
	\IfEqCase{#1}{
		{4}{
			\begin{table}[H]
				\centering
				\begin{NiceTabular}[width=\linewidth]{@{}!{\qquad}*{4}{X[1,l]}}
					A. #2 & B. #3 & C. #4 & D. #5
				\end{NiceTabular}
			\end{table}
		}
		{2}{
			\begin{table}[H]
				\centering
				\begin{NiceTabular}[width=\linewidth]{@{}!{\qquad}*{2}{X[1,l]}}
					A. #2 & B. #3 \\
					C. #4 & D. #5
				\end{NiceTabular}
			\end{table}
		}
		{1}{
			\begin{table}[H]
				\centering
				\begin{NiceTabular}[width=\linewidth]{@{}!{\qquad}X[1,l]}
					A. #2 \\
					B. #3 \\
					C. #4 \\
					D. #5
				\end{NiceTabular}
			\end{table}
		}
	}
}
\NewDocumentCommand{\blank}{}{\underline{\qquad}}

\title{北京邮电大学2023-2024学年第二学期\\《数学分析(下)》期末考试试题(A卷)}
\author{重制:cppHusky}
\date{}
\begin{document}
\pagestyle{fancy}
\fancyhf{}
\fancyhead[L]{\footnotesize\sffamily 23-24第二学期《数学分析(下)期末A卷}
\fancyhead[R]{\footnotesize\sffamily \textit{cppHusky}重制}
\fancyfoot[C]{\footnotesize\sffamily 第\thepage 页,共\pageref*{LastPage} 页}
\fancyfoot[R]{\footnotesize \href{https://byrdocs.org/}{byrdocs.org}}
\maketitle
\thispagestyle{fancy}
\section{填空题 (本大题共10小题, 篛小题3分, 共30分)}
\subsection{级数$\sum_{n=1}^{\infty}\frac{\ln{n}}{\sqrt{n}}$的敛散性为\blank(收敛或发散).}
\subsection{幂级数$\sum_{n=1}^{\infty}\qty(\sqrt{n+1}-\sqrt{n})2^{n}x^{2n}$的收敛区间为\blank.}
\subsection{已知$f(x,y)=\begin{cases}
	\frac{\ln(1+xy)}{xy},&xy\ne0\\
	x,&xy=0
\end{cases}$, 则$f_y(1,0)$=\blank.}
\subsection{已知函数$z=z(x,y)$由方程$xy+z+\ee^{xz}=1$所确定, 则$\eval{\dd{z}}_{(1,0)}=$\blank.}
\subsection{曲面$z=\arctan{\frac{y}{x}}$在点$\qty(1,1,\frac{\pi}{4})$处的切平面方程为\blank.}
\subsection{点$(0,0)$是函数$f(x,y)=(1+\ee^{y})\cos{x}-y\ee^{y}$的\blank(极大或极小) 值点.}
\subsection{设平面区域$D$由曲线$y=x^{3}$与直线$x=1,\,y=-1$围成, 则$\iint_{D}\qty[1+xy\ee^{\abs{x}+\abs{y}}]\dd{x}\dd{y}=$\blank.}
\subsection{已知$D$是第一象限中曲线$y=\sqrt{1-x^{2}}$与直线$x+y=1,\,y=x,\,y=\sqrt{3}x$围成的平面区域, $f(t)$连续, 将二重积分$\iint_{D}f(x^{2}+y^{2})\dd{\sigma}$化为极坐标系下的累次积分为\blank.}
\subsection{设闭曲线$C$为$\abs{x}+\abs{y}=1$, 则$\oint_{C}\qty(\abs{x}+y)\dd{s}=$\blank.}
\pagebreak
\subsection{已知$C$为不经过原点的一条光滑曲线, 从点$A(1,0)$到点$B(3,4)$, 则\newline$\int_{C}\frac{x\dd{x}+y\dd{y}}{\sqrt{x^{2}+y^{2}}}=$\blank.}
\section{(10分) 设$f(r)$有二阶连续导数,$z=z(x,y)=f(r)$满足方程$\pdv[2]{z}{x}+\pdv[2]{z}{y}=0$, 其中$r=\sqrt{x^{2}+y^{2}}$. 若$f(1)=0,\,f'(1)=1$, 求$z=z(x,y)$的表达式.}
\section{(12分)}
\subsection{计算累次积分$\int_{1}^{2}\dd{y}\int_{\sqrt{y}}^{y}\ee^{\frac{y}{x}}\dd{x}+\int_{2}^{4}\dd{y}\int_{\sqrt{y}}^{2}\ee^{\frac{y}{x}}\dd{x}$.}
\subsection{已知$\Omega$是由圆锥面$z=\sqrt{x^{2}+y^{2}}$和上半球面$z=\sqrt{1-x^{2}+y^{2}}$围成的闭区域, 计算三重积分$\iiint_{\Omega}z(x^{2}+y^{2})\dd{x}\dd{y}\dd{z}$.}
\section{(12分) 设$a,\,b$为实数, 函数$z=ax^{2}+by^{2}+2$在点$(3,4)$的方向导数中, 沿方向$\overrightarrow{l}=-3\overrightarrow{i}-4\overrightarrow{j}$的方向导数最大, 最大值为$10$.}
\subsection{求$a,\,b$的值.}
\subsection{求曲面$S:\:z=ax^{2}+by^{2}+2\;(z\ge0)$的面积.}
\section{(12分) 设$C$是原点$O(0,0,0)$为起点, 椭球面$\frac{x^{2}}{a^{2}}+\frac{y^{2}}{b^{2}}+\frac{z^{2}}{c^{2}}=1$上第一卦限内的点$(x_0,y_0,z_0)$为终点的有向直线段,问当$x_0,\,y_0,\,z_0$为何值时, 曲线积分$I=\int_{C}yz\dd{x}+zx\dd{y}+xy\dd{z}$的值最大, 并求出最大值.}
\section{(10分) 设$C$是圆周$(x-a)^{2}+(y-a)^{2}=1$的边界曲线, 逆时针方向, $f(x)$为大于零的连续函数, 证明: $\oint_{C}\qty[\ee^{x}\cos{y}-\frac{y}{f(x)}]\dd{x}+\qty[xf(y)-\ee^{x}\sin{y}]\dd{y}\ge2\pi$.}
\section{(10分) 已知$S$是上半球面$z=\sqrt{a^{2}-x^{2}-y^{2}}\;(a>0)$的上侧, 计算曲面积分$I=\iint_{S}ax\dd{y}\dd{z}-3ay\dd{z}\dd{x}+(z+a)^{2}\dd{x}\dd{y}$.}
\end{document}
